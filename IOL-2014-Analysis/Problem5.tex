\section{第五题}

以下为西北格巴亚语的一些词和词组及其汉语翻译(乱序排列):

\begin{center}
\bipa{ʔáá, ʔáá náng nú kò, ʔáá sèè, búmá yík, búmá zù yík, dáng gòk, dí fò, dí sèè,\\
kò yík, kò zòk, náng wí, nú fò, nú lébé, sèè wí, yík, yík wí, zù}
\medskip

在……的表面; 眼窝; 眉毛; 睫毛; 眼睛/脸;\\
田边; 脚; 幸福; 肝; 良田; 鼻孔;\\
在……的上方; 毒蛇; 舌尖;\\
死 (动词); 嫉妒 (动词); 放 (动词)

\end{center}

\begin{assgts}
\item 将原文与译文正确对应起来.
\item 翻译成汉语: \bipa{búmá zù, kò, lébé gòk, lébé wí}.
\item 翻译成西北格巴亚语: 在……的中央; 头; 不愉快; 鼻子.
\end{assgts}

西北格巴亚语属于乌班吉语系。在中非共和国,尚有二十万人使用这门语言。

第五题是最难的一道题。本题的最佳解题奖给了来自上海外国语大学附属外国语学校的陈天陆,文科生。
她同时也是本届比赛中得分最高的中国队队员,以及 IOL 2014 Jeopardy 总分第三名。想要勾搭的同学们,加油!

鉴于我自己并没有认真做此题,所写的推理过程必受答案影响,我便邀请陈天陆分享她的推理过程,也让大家感受一下大神的气息。

\subsection{陈小姐的脑洞}
事先声明,陈天陆先将她的推理过程写在纸上,再拍下来发给我。我不得不手工完成再录入工作。与原稿相比,
文档布局基本不变,增加直角引号,重绘两个统计表格,并出于美观和卖萌考虑,将一处 “xx” 改为\sq{啥}。

如果大家感觉完全看不懂,或者太跳跃了,我在后面还附上了我精心构造的推理——虽然我是看过答案才写的。

\subsubsection{统计}
\begin{tabular}{c|c|l}
\hline
词数 & 次数 \\
\hline
1 & 3 & \word{ʔáá, yík, zù} \\
2 & 13 & 略 \\
3 & 1 & \word{búmá zù yík} \\
4 & 1 & \word{ʔáá náng nú kò} \\
\hline
\end{tabular}

\begin{tabular}{l|l}
\hline
次数 & 单词 \\
\hline
5 & \word{yík} \\
4 & \symfont{∅} \\
3 & \word{ʔáá, nú, kò, wí} \\
2 & \word{náng, sèè, búmá, zù, dí, fò} \\
1 & \word{dáng, gòk, sèè, zòk, lébé} \\
\hline
\end{tabular}

\subsubsection{猜}
最先猜出 \word{yík} =\sq{眼},因为出现频率最高,于是 \word{búmá zù yík} 和 \word{búmá yík} 分别对应睫毛和眉毛。
猜测\sq{睫毛}=\sq{眼的毛},\sq{眉毛}=\sq{眼上的毛},于是 \word{zù} 为\sq{在……的上方},
\word{kò yík} 是\sq{眼窝},\word{kò} =\sq{洞},\word{zòk} =\sq{鼻}。

剩下 \word{yík} 和 \word{yík wí},其中一个是\sq{眼}。考虑到:一、介词一般是单个词;
二、\sq{脸}$\approx$\sq{表面},\word{yík} =\sq{在……的表面},\word{wí} =\sq{身体部分}。
由此 \word{sèè wí} 和 \word{náng wí} 也是身体部分,推测一个是\sq{脚}一个是\sq{肝}(\sq{舌尖}不够基本)。

这时发现:

\newcommand \ctltable [3]{#1 = #2 & & #3 }
\begin{tabular}{rlll}
A & 的 & B & \\
\ctltable{幸福}{好}{肝} & 
\multirow{2}{10cm}{(还有一个原因是下题中要翻译\sq{在……中央}和\sq{不愉快},这让人联想到\sq{肝}是个重要的词)} \\
\ctltable{嫉妒}{啥}{肝} & \\
\ctltable{良田}{好}{田} &
\multirow{3}{10cm}{反正这就确定\sq{肝}= \word{sèè wí},那\sq{在……的中央}= \word{sèè},\word{náng} =\sq{脚}} \\
\ctltable{良田}{田}{边} & \\
\ctltable{舌尖}{舌}{边} & \\
\end{tabular}

这时候基本确定\sq{舌头} = \word{lébé},\word{dí} =\sq{好},\word{fò} =\sq{田},\word{nú}=\sq{边}。

然后看 \word{ʔáá náng nú kò},后三个词是\sq{脚+边+洞},听上去十分凶险,应该是\sq{死}……

不知道 \word{ʔáá} 和 \word{dáng gòk} 哪个是\sq{毒蛇},好像都解释得通,但下面有 \word{lébé gòk},
因为词能拆开,\word{dáng gòk} 应是 \sq{毒蛇},至此对应完毕。

\word{búmá zù} = 上面的毛 = 头发\\
\word{lébé gòk} = 蛇舌\\
\word{lébé wí} = 舌

头 = \word{zù}\\
不愉快 = 毒肝 = \word{dáng sèè}\\
鼻 = \word{zòk wí}

\subsection{我编出来的推理}

先看汉语翻译。\sq{眼窝}、\sq{眉毛}和\sq{睫毛},都与\sq{眼睛}有关。
翻译中还出现了\sq{在……之上},而眉毛,就在眼睛的上方。因此,这是一个很好的突破口:

\begin{tabular}{l|l}
\hline
次数 & 单词 \\
\hline
5 & \word{yík} \\
4 & \symfont{∅} \\
3 & \word{ʔáá, nú, kò, wí} \\
2 & \word{náng, sèè, búmá, zù, dí, fò} \\
1 & \word{dáng, gòk, sèè, zòk, lébé} \\
\hline
\end{tabular}

只有 \word{yík} 出现超过四次,必然与眼睛有关。

\begin{center}
\word{búmá yík}, \word{búmá zù yík}, \word{kò yík}, \word{yík wí}, \word{yík}
\end{center}

有 \word{búmá} 的两个短语应当为\sq{眉毛}和\sq{睫毛}。\sq{睫毛}是眼睛的毛,
而\sq{眉毛}是眼睛上方的毛,恰好 \word{zù} 独立出现过,翻译里也有\sq{在……的上方}。
因此,前两个词分别为\sq{睫毛}和\sq{眉毛}。\word{zù} 是\sq{在……的上方}。

可以推断,\word{A B} 在西北格巴亚语中类似 “B 的 A”,因此 \word{kò yík} 是\sq{眼窝},
\word{kò} 可能是\sq{洞}(\sq{鼻孔}),也可能是\sq{……的边缘}(\sq{田边})。

\word{kò} 出现在词首只有两次,另一次是 \word{kò zòk},若其代表\sq{……的边缘},
则 \word{zòk} 代表\sq{田},可后者只出现过一次,而\sq{田}还在\sq{良田}中出现过。
因此,\word{kò} 是\sq{洞},\word{kò zòk} 是\sq{鼻孔}。

回到 \word{yík wí},\word{wí} 这词还出现了两次,均在词尾。尚未确定的器官还有
\sq{肝}、\sq{脚} 和 \sq{舌尖}。\sq{舌尖} 看上去更是一个复合词(\sq{在舌头的顶端}),
所以,\word{wí} 位于词尾可能表示基本器官。实际上,\word{wí} 是\sq{人},
不过这对解题没有影响。

现在我们还剩下以下三组词组未经讨论:

\begin{enumerate}
\item \word{\hlc{ʔáá}, \hlc{ʔáá} náng nú kò, \hlc{ʔáá} sèè}

\item \word{\hlb{dí} \hla{fò}, \hlb{dí} sèè, \hlb{nú} \hla{fò}, \hlb{nú} lébé}

\item \word{dáng gòk}
\end{enumerate}

第一组均以 \word{\hlc{ʔáá}} 开头。第二组之间相互关联紧密。第三组……比较孤立。

假设第一组对应三个动词,理由如下:一个派生出来的动词,其构成成分中至少含有一个动词成分,
如\sq{快走}里的\sq{走},或者桑戈语 \word{gi nyama}(\sq{打猎})里的 \word{gi}(\sq{搜寻})
\cite{NOL2014Round1Problem1}。

第二组词相互之间的关系,和四个翻译极为相似,如下图所示:

\SetVertexNormal[
  Shape=rectangle,
  FillColor=white,
  LineWidth=1pt]
\tikzset{VertexStyle/.append style={font=\bfseries\ipafont}}
\SetUpEdge[
  lw=1pt,
  color=black,
  labelcolor=white,
  labelstyle={font=\bfseries\ipafont}]
\begin{tikzpicture}
  \Vertex[x=0,   y=0]{sèè}
  \Vertex[x=1.5, y=4]{dí}
  \Vertex[x=3,   y=0]{fò}
  \Vertex[x=4.5, y=4]{nú}
  \Vertex[x=6,   y=0]{lébé}
  \tikzset{EdgeStyle/.style={->}}
  \Edge[label=\word{dí sèè}](dí)(sèè)
  \Edge[label=\word{dí fò}](dí)(fò)
  \Edge[label=\word{nú fò}](nú)(fò)
  \Edge[label=\word{nú lébé}](nú)(lébé)
\end{tikzpicture}
\quad
\tikzset{VertexStyle/.append style={font=\normalfont}}
\SetUpEdge[
  lw=1pt,
  color=black,
  labelcolor=white,
  labelstyle={font=\normalfont}]
\begin{tikzpicture}
  \Vertex[x=0,   y=0]{心情}
  \Vertex[x=1.5, y=4]{好}
  \Vertex[x=3,   y=0]{田}
  \Vertex[x=4.5, y=4]{边}
  \Vertex[x=6,   y=0]{舌头}
  \tikzset{EdgeStyle/.style={->}}
  \Edge[label=幸福](好)(心情)
  \Edge[label=良田](好)(田)
  \Edge[label=田边](边)(田)
  \Edge[label=舌尖](边)(舌头)
\end{tikzpicture}

由于这个关系是完全对称的,我们只能确定 \word{fò} 就是 \sq{田}。但这进一步肯定了前面关于动词的猜想。
剩下的 \word{dáng gòk} 便是\sq{毒蛇}。

注意 \word{dí sèè}。前面讨论过,\word{sèè wí} 代表器官\sq{肝}或\sq{脚}。
我们知道,类似心脏这样的内脏器官,在很多语言中表示心情,如\sq{心情}。因此,
\word{dí sèè} 应该与心情有关,即\sq{幸福},\word{sèè wí} 是\sq{肝},
\word{náng wí} 是\sq{脚}。剩下的词也就随之确定了:

\begin{tabular}{c|c}
\hline
词组 & 翻译 \\
\hline
\word{dí} & \sq{好} \\
\word{nú} & \sq{……的边缘} \\
\word{dí fò} & \sq{良田} \\
\word{nú fò} & \sq{田边} \\
\word{nú lébé} & \sq{舌尖} \\
\hline
\end{tabular}

回到动词。\sq{放}在三个动词最为原始,我猜它就是 \word{ʔáá},代入:

\begin{tabular}{c|c|c}
\hline
词组 & 直译 & 翻译 \\
\hline
\word{ʔáá náng nú kò} & 放脚于洞边 & 死 \\
\word{ʔáá sèè} & 放肝 & 嫉妒 \\
\hline
\end{tabular}

还剩一词,即最初的突破口 \word{yík},便是\sq{在……的表面}了。

解决了两道小题的读者会发现,在西北格巴亚语中,方位词和器官词一致,许多非洲语言都有这一特征。

