\section{第五题}

以下为西北格巴亚语的一些词和词组及其汉语翻译(乱序排列):

\begin{center}
\bipa{ʔáá, ʔáá náng nú kò, ʔáá sèè, búmá yík, búmá zù yík, dáng gòk, dí fò, dí sèè,\\
kò yík, kò zòk, náng wí, nú fò, nú lébé, sèè wí, yík, yík wí, zù}
\medskip

在……的表面; 眼窝; 眉毛; 睫毛; 眼睛/脸;\\
田边; 脚; 幸福; 肝; 良田; 鼻孔;\\
在……的上方; 毒蛇; 舌尖;\\
死 (动词); 嫉妒 (动词); 放 (动词)

\end{center}

\begin{assgts}
\item 将原文与译文正确对应起来.
\item 翻译成汉语: \bipa{búmá zù, kò, lébé gòk, lébé wí}.
\item 翻译成西北格巴亚语: 在……的中央; 头; 不愉快; 鼻子.
\end{assgts}

西北格巴亚语属于乌班吉语系。在中非共和国,尚有二十万人使用这门语言。

第五题是最难的一道题。本题的最佳解题奖给了来自上海外国语大学附属外国语学校的陈天陆,文科生。
她同时也是本届比赛中得分最高的中国队队员,以及 IOL 2014 Jeopardy 总分第三名。想要勾搭的同学们,加油!

鉴于我自己并没有认真做此题,所写的推理过程必受答案影响,我便邀请陈天陆分享她的推理过程,也让大家感受一下大神的气息。

\subsection{陈小姐的脑洞}
事先声明,陈天陆先将她的推理过程写在纸上,再拍下来发给我。我不得不手工完成再录入工作。与原稿相比,
文档布局基本不变,增加直角引号,重绘两个统计表格,并出于美观和卖萌考虑,将一处 “xx” 改为\sq{啥}。

如果大家感觉完全看不懂,或者太跳跃了,我在后面还附上了我精心构造的推理——虽然我是看过答案才写的。

\subsubsection{统计}
\begin{tabular}{c|c|l}
\hline
词数 & 次数 \\
\hline
1 & 3 & \word{ʔáá, yík, zù} \\
2 & 13 & 略 \\
3 & 1 & \word{búmá zù yík} \\
4 & 1 & \word{ʔáá náng nú kò} \\
\hline
\end{tabular}

\begin{tabular}{l|l}
\hline
次数 & 单词 \\
\hline
5 & \word{yík} \\
4 & \symfont{∅} \\
3 & \word{ʔáá, nú, kò, wí} \\
2 & \word{náng, sèè, búmá, zù, dí, fò} \\
1 & \word{dáng, gòk, sèè, zòk, lébé} \\
\hline
\end{tabular}

\subsubsection{猜}
最先猜出 \word{yík} =\sq{眼},因为出现频率最高,于是 \word{búmá zù yík} 和 \word{búmá yík} 分别对应睫毛和眉毛。
猜测\sq{睫毛}=\sq{眼的毛},\sq{眉毛}=\sq{眼上的毛},于是 \word{zù} 为\sq{在……的上方},
\word{kò yík} 是\sq{眼窝},\word{kò} =\sq{洞},\word{zòk} =\sq{鼻}。

剩下 \word{yík} 和 \word{yík wí},其中一个是\sq{眼}。考虑到:一、介词一般是单个词;
二、\sq{脸}$\approx$\sq{表面},\word{yík} =\sq{在……的表面},\word{wí} =\sq{身体部分}。
由此 \word{sèè wí} 和 \word{náng wí} 也是身体部分,推测一个是\sq{脚}一个是\sq{肝}(\sq{舌尖}不够基本)。

这时发现:

\newcommand \ctltable [3]{#1 = #2 & & #3 }
\begin{tabular}{rlll}
A & 的 & B & \\
\ctltable{幸福}{好}{肝} & 
\multirow{2}{10cm}{(还有一个原因是下题中要翻译\sq{在……中央}和\sq{不愉快},这让人联想到\sq{肝}是个重要的词)} \\
\ctltable{嫉妒}{啥}{肝} & \\
\ctltable{良田}{好}{田} &
\multirow{3}{10cm}{反正这就确定\sq{肝}= \word{sèè wí},那\sq{在……的中央}= \word{sèè},\word{náng} =\sq{脚}} \\
\ctltable{良田}{田}{边} & \\
\ctltable{舌尖}{舌}{边} & \\
\end{tabular}

这时候基本确定\sq{舌头} = \word{lébé},\word{dí} =\sq{好},\word{fò} =\sq{田},\word{nú}=\sq{边}。

然后看 \word{ʔáá náng nú kò},后三个词是\sq{脚+边+洞},听上去十分凶险,应该是\sq{死}……

不知道 \word{ʔáá} 和 \word{dáng gòk} 哪个是\sq{毒蛇},好像都解释得通,但下面有 \word{lébé gòk},
因为词能拆开,\word{dáng gòk} 应是 \sq{毒蛇},至此对应完毕。

\word{búmá zù} = 上面的毛 = 头发\\
\word{lébé gòk} = 蛇舌\\
\word{lébé wí} = 舌

头 = \word{zù}\\
不愉快 = 毒肝 = \word{dáng sèè}\\
鼻 = \word{zòk wí}

\subsection{我编出来的推理}

先看汉语。\sq{眼窝}、\sq{眉毛}、\sq{睫毛}和\sq{眼睛},都是和眼睛有关的东西。
汉语中还出现了\sq{在……的表面}与\sq{在……之上},而眉毛,就在眼睛的上方。
因此,这是一个很好的突破口。

\begin{tabular}{l|l}
\hline
次数 & 单词 \\
\hline
5 & \word{yík} \\
4 & \symfont{∅} \\
3 & \word{ʔáá, nú, kò, wí} \\
2 & \word{náng, sèè, búmá, zù, dí, fò} \\
1 & \word{dáng, gòk, sèè, zòk, lébé} \\
\hline
\end{tabular}

可以看到,出现了四次及以上的单词只有 \word{yík} 一个,其必然与眼睛有关。五个由其组成的短语中,
\word{yík} 位于词尾的有三个,即 \word{búmá yík}、\word{búmá zù yík} 和 \word{kò yík},
剩下为 \word{yík wí} 和 \word{yík}。鉴于 \word{búmá} 出现两次,可以推出其与毛有关,
构成的两个短语为\sq{眉毛}和\sq{睫毛},\sq{睫毛} 是眼睛的毛,\sq{眉毛} 在眼睛上方,
\word{zù} 也恰好独立出现过一次,应该就是\sq{在……的上方}。

亦可看出,\word{A B} 在西北格巴亚语中类似 “B 的 A”,因此 \word{kò yík} 是\sq{眼窝}(\sq{鼻孔}),
\word{kò} 可能是\sq{洞},也可能是\sq{在……的边上}(\sq{田边},眼窝可以看作眼球立体的边界)。

\word{kò} 出现在词首只有两次,另一次是 \word{kò zòk},若其代表\sq{在……的边上},
则 \word{zòk} 代表\sq{田},可后者只出现过一次,而\sq{田}还在\sq{良田}中出现过。
因此,\word{kò} 是\sq{洞},\word{kò zòk} 是\sq{鼻孔}。

回到 \word{yík wí},\word{wí} 这词还出现了两次,均在词尾。尚未确定的器官还有
\sq{肝}、\sq{脚} 和 \sq{舌尖}。\sq{舌尖} 看上去更是一个复合词,
因此 \word{wí} 位于词尾可能表示独立的器官。

现在我们还剩下以下词组:

\begin{enumerate}
\item \word{\hlc{ʔáá}, \hlc{ʔáá} náng nú kò, \hlc{ʔáá} sèè}

\item \word{\hlb{dí} \hla{fò}, \hlb{dí} sèè, \hlb{nú} \hla{fò}, \hlb{nú} lébé}

\item \word{dáng gòk}
\end{enumerate}

较为复杂的动词或动词短语,其语素或构成成分,至少有一个动词成分(词缀或中心词),如\sq{快走}里的\sq{走},
以及第三届全国语言学奥赛初赛第一题。这个动词成分的位置一般固定,如\sq{快走}和\sq{慢走}。考虑到 \word{ʔáá} 出现了三次,
均出现在词首,有理由相信这三个词对应着三个动词。

第二组词组中,\hlb{\word{dí}} 和 \hlb{\word{nú}} 各出现两次,均出现在词首。\hla{\word{fò}} 将这两类词连接起来。
对中文翻译分类,可以将\sq{幸福}和\sq{良田}分为一组(与\sq{好}有关),将\sq{田边}和\sq{舌尖}分为一组
(与\sq{在……的边上}有关),\sq{田}将这两类词连接起来。有理由相信,\hla{\word{fò}} 就是 \sq{田},我们只需判断,
\hlb{\word{dí}} 和 \hlb{\word{nú}},谁是\sq{好},谁是\sq{在……的边上}。

\word{\hlb{dí} sèè} 是一个引人注目的词,\word{sèè} 与 \word{wí} 构成一个短语,尚不知是\sq{肝}还是\sq{脚}。
我们知道,内脏,如\sq{心},在很多语言中表示心情。因此 \word{\hlb{dí} sèè} 应当就是\sq{幸福},
\word{sèè wí} 是\sq{肝},\word{náng wí} 是\sq{脚}。剩下的词也就随之确定了:

\begin{tabular}{c|c}
\hline
词组 & 翻译 \\
\hline
\word{dí} & \sq{好} \\
\word{nú} & \sq{在……的边上} \\
\word{dí fò} & \sq{良田} \\
\word{nú fò} & \sq{田边} \\
\word{nú lébé} & \sq{舌尖} \\
\word{dáng gòk} & \sq{毒蛇} \\
\hline
\end{tabular}

回到动词。\sq{放}在三个动词最为原始,我猜它就是 \word{ʔáá},这样一来:

\begin{tabular}{c|c|c}
\hline
词组 & 直译 & 翻译 \\
\hline
\word{ʔáá náng nú kò} & 放脚于洞边 & 死 \\
\word{ʔáá sèè} & 放肝 & 嫉妒 \\
\hline
\end{tabular}

\word{yík} 是\sq{在……的表面}。解决了两道小题的读者会发现,在西北格巴亚语中,方位词和器官词一致,
独立表示器官时需要加上 \word{wí},后者意为\sq{人}。许多非洲语言都有这一特征。
